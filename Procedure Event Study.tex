\documentclass[12pt]{article}
	\usepackage[T1]{fontenc}
	\usepackage[utf8]{inputenc}
	\usepackage[british]{babel}
	\usepackage[a4paper]{geometry}
	\geometry{verbose,tmargin=3cm,bmargin=3.5cm,lmargin=4cm,rmargin=3cm,marginparwidth=70pt}
	\setcounter{secnumdepth}{3}
	\setcounter{tocdepth}{3}
	\usepackage{prettyref}
	\usepackage{textcomp}
	\usepackage{setspace}
	\usepackage{indentfirst}
	\usepackage{fancyhdr}
	\usepackage{url}
	\usepackage[normalem]{ulem}
	\usepackage[table, fixpdftex]{xcolor}
	\usepackage{algpseudocode}
	\usepackage{bigstrut}
	\usepackage{enumitem}

	% package hyperref
	\usepackage{hyperref}
	
	% fancy headers for the thesis
	\fancyhead{}
	\fancyhead[RO]{\slshape \nouppercase \rightmark}
	\fancyfoot[OC]{\begin{flushright}\thepage\end{flushright}}
	\renewcommand{\headrulewidth}{0.4pt}
	\setlength{\headheight}{14pt}

\title{Event-Study}


\author{Leopold Ingenohl}


\begin{document}
\maketitle

\section{General Procedure}


\begin{enumerate}
	\item Event Window (Aim) - The initial task is to define the event of interest and identify the period over which the security prices of the firms involved in this event will be examined
	\item Selection Criteria Sample - restrictions imposed on the firms involved in the sample.
	\item Summarize Sample Characteristics - market capitalization, industry representation, distribution of events through time
	\item Market Model - This normal performance model assumes a stable linear relation between the market return and the security return
	\item Estimation Window - using the period prior to the event window (exclusion of event window in the estimation window) for computing the normal return
	\item Abnormal returns - Difference between the actual return and the normal return 
	\item Design Testing Framework - Defining the null hypothesis and determining the techniques for aggregating the individual firm abnormal returns
	\item Formulation Econometric Design - presentation of diagnostics (limited number of observations etc.)
	\item 
\end{enumerate}

\section{Results}

\begin{enumerate}
	\item Graphics on the abnormal returns for different groups of firms 
	\item Cross-Sectional Models - examining the association between the magnitude of the abnormal return and the characteristics specific to the event observation (Company condition)
	\item show, that the announcement has a significant effect on the stock price (statistical test) - Relevance of 13D Filings for the analysis (quickly show with literature)
\end{enumerate}

\section{Models for Measuring Normal Performance}
\begin{enumerate}
	\item 
\end{enumerate}

\section{Comments}

\begin{itemize}
	\item Probably no clustering of events (The date of each filing only affects one company)? tbd
\end{itemize}

\end{document}
